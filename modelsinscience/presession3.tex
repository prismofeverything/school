\title{Worldview Modeling - Week 3 Pre-Session Assignment}
\author{Ryan Spangler}
\date{\today}

\documentclass[12pt]{article}

\usepackage{graphicx}

\setcounter{secnumdepth}{0}

\begin{document}
\maketitle

\section{Question 1}

\emph{Gilbert’s description of “simulation models” is a good description of scientific models in general.  What are simulation models used for?}

\vspace{4pt}

Simulation models are used when it is difficult or impossible to do experiments on the target itself, or there is a large risk or cost of doing so.  Many systems are like this: stock markets, human populations, oceans... really many interesting things are prohibitive to study directly.  The simulation acts as a surrogate to the actual phenomenon in question, allowing a wide range of conditions to be imposed and conclusions to be drawn without affecting the actual subject of inquiry.

\section{Question 2}

\emph{Compare and contrast statistical models and simulation models.}

\vspace{4pt}

Statistical and simulation models are both models, so are both systems we use to mimick the behavior and form predictions of the original system.  Statistical models are primarily about correlating and working with the available data to find some statistical means to describe and reproduce various effects of the system.  Simulation models on the other hand attempt to bring forth the dynamics of the original system by capturing and abstracting the mechanisms that generate the behavior in the first place.  In this way simulation models attempt to be more explanatory in general by looking at how the behavior of the target comes about, where statistical models are more teleological and describe what is observed, but not as much why.  

\section{Question 3}

\emph{List and describe the stages of simulation-based research.}

\vspace{4pt}

The first step is defining the question:  what are we actually studying here?  Next you actually have to build a model of the system you are inquiring about.  Once a model is built, you have to verify that it works correctly, and does not contain corner cases or bugs where it departs wildly from what you expect.  After verification, the new model has to actually be compared to real world dynamics in order to see how effective the model was in actually capturing the target behavior.  All throughout this process any of the stages may have to be revisited.  Once the model has been defined, built, verified and validated then you can publish some results, and achieve fame and fortune and immortality.

\section{Question 4}

\emph{Drawing from the conclusion and your own imagination, what strengths and weaknesses might there be in using models for scientific inquiry?}

\vspace{4pt}

The main advantage to using models for scientific inquiry is the ability to ask questions and impose alternative and extreme conditions on the target without affecting the target itself.  Also, the abstraction of the process into a formal model provides the opportunity to transform the model independently of any real-world interpretation using mathematical or computational methods that could give novel insight into the system that would not have been discovered by considering the model in its natural setting.

\section{Question 5}

\emph{Write a Discussion Question from this material that you could share with the class.}

\vspace{4pt}

How can you tell when your model is close enough to the target behavior to be useful?  Where do you draw the line?

\end{document}
