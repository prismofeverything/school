
PS I put an example down here of one of the failings (in my mind) of pure logic.  

Let me give a (hopefully) clarifying example.  Here is a simple logical argument:

  --> Dogs have fleas
  --> George is a dog
  --> George has fleas

Logic is all about making these truth-preserving transformations.  Cool.  But why can't we say Dogs have George?  Or fleas are a George?  This makes immediate (non)sense to us, but turns out to be something that logic doesn't have anything to say about.  It is not wrong, it is just agnostic to things that aren't part of a linear (there's that first order part) chain of implication.  All other structure is removed.  Everyone just assumes this mapping and goes about mashing the world into logical statements, which can be misleading if you aren't *extremely* careful about what you are doing.  Regardless, I hope at some point you can see the most complex part of these statements has nothing to do with logic.

Understanding what a flea is, how it relates to other animals, its existence, its "fleaness", is not something we understand yet.  Because we don't understand how we understand even the simplest thing like a flea, much less something as elusive and fundamental as "charge", or the perennial favorite, "time", it is hard to c













































































Hello all,

I could tell Rajesh was disappointed I refused to accept the most basic assumption of Simon's work, and our time together is precious, so in the interest of avoiding another seminar where I rant on and on about this maybe we can hash this out over the list and move on to something else in class orb.































My main issue was about this distinction between scientific and "regular" knowledge.  Not that I am opposed to making such a distinction, I just find several things troubling about how to really go about making this distinction.  First off, I think some important context is that the reason Simon is making this distinction is because he wants to automate the discovery of scientific knowledge.  So for him it is important to make this distinction because he predicts in some way scientific knowledge will be easier to *formalize* than general knowledge, that somehow its structure is more amenable to casting into an automatable process which can then be used to generate knowledge without the need for human intervention.

The first issue I have is the concept of knowledge itself is vaguely defined.  I checked around and philosophers have *still* not really figured out what knowledge is.  For centuries now it is all they have been arguing about, and there is still no consensus.  There is not much progress really from Plato's original JTB (justified true belief) idea.  So this is already shaky ground, but I think we all agreed that knowledge is more than data, it is more than just bits.  It is also the relationship between different known things, and in a way any part of a body of knowledge is inextricable from the whole in which it is embedded.  You can't know that grass is green without knowing what grass is and what green looks like.  Somehow that ``justified'' and ``belief'' part work their way in and turn the whole thing into a mess that can't really be separated from cognition itself.  

So, what makes scientific knowledge distinct from ``other'' knowledge?  What really?  Someone mentioned falsifiability, I think we can throw some other things in there like ``based on observation and experiment'', ``using reason or formal processes'', all the properties we know and love from science.    
