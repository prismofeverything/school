Ryan Spangler

Software Engineer
 
I can build anything.

Languages:
C, C++, Python, Ruby, Matlab, SQL, Javascript, Actionscript, Bash, Perl, Java, Clojure, Scheme, etc. (and I can pick up anything at this point)

Things I have built:

A series of AppEngine sites for Google:
  Chrome Experiments: (http://chromeexperiments.com) The center for innovative javascript experiments on the web.
  What Browser: (http://whatbrowser.org) An educational site for people who don't know what a web browser is!

Interface CMS (http://getinterface.com):  A CMS I built to be a support infrastructure for a series of diverse sites, solving each problem in a way so that all the sites, and eventually services, could benefit from it.  Flexibility paired with a clean conceptual approach.  Later it became a product in its own right!

Here are some of the sites that I helped develop while improving Interface:
  Uncommon: (http://getuncommon.com) Customize your own iphone case with any image you want.
  Rethink What Matters (Bare Escentuals): (http://rethinkwhatmatters.com) A quiz and contest site for the popular cosmetic company.
  Ziba: (http://ziba.com) A tasteful and subtly technical site for the local design tribe.
  Giro: (http://giro.com) Basic product site for the bicycle helmetcompany.
  Salomon Snowboards: (http://salomonsnowboard.com)
  Portland Monthly Magazine: (http://portlandmonthlymag.com) A huge magazine site with calendar, map and subscription functionality.
  Gridplane: (http://gridplane.com) Local design by JD Hooge.
  Instrument: (http://weareinstrument.com)
  Invisible Creature: (http://invisiblecreature.com)
  ... and many more (see the getinterface.com site for more examples).

Some of the challenges I have conquered:
  Designing an endlessly flexible framework for defining models and associations that could take on a specific form custom tailored to the data entailed by each site.
  Designing a compact query language to fetch data meeting particular criteria across associations, and translating those queries to raw, ruthlessly optimized SQL.
  Server scaling and the implementation of an nginx module for hashing image queries to cache images down to the bare metal. 

Employment

October 2007-Present  Senior Software Engineer, Instrument, Portland OR.  At Instrument I have run the whole gamut of roles and duties, doing anything and everything no one else could do.  By building a generalized framework, I automated many of the programming tasks and duties that had to be performed manually before.  

May 2006-October 2007  Software Engineer, Performance Logic, Portland OR.  I was hired by this company because they had a bug list 600 items long.  In the course of my time there I took a substantial portion of the things they were doing fifteen different ways in fifteen different places and gathered them into a sensible, coherent system.


Internship

May 2000-August 2000  Undergraduate Research Software Engineer, Rochester Institute of Technology, Rochester NY.  Designed and implemented an interactive modular synthesis program in Java. 


Education

Sept 2009-Today  Portland State University, Portland OR
Masters in Systems Science

Sept 2002-June 2005  The Evergreen State College, Olympia WA
BA in math, performance and computer science.

Jan 1999-Jan 2001  Oberlin College, Oberlin OH
focus in computation and cognitive science.


Personal Projects and Web Art:
  Homeostasis (http://youdonotexist.com/homeostasis): An interactive simulation of one circuit in the perceptual/metabolic pathway of a cell.
  You Do Not Exist (http://youdonotexist.com): Challenging the assumption that you actually exist.
  Zarathustra Speaks (http://zarathustraspeaks.com): Markov-generated wisdom from the text of "Thus Spake Zarathustra" by Frederich Nietzsche.
  Spume (http://spu.me): An abstract visualization of an idealized world.

I thrive on challenge and live to simplify seemingly impossible computational tasks.  Recommendation engines to graph algorithms, language parsing to computational biology, the harder the better.  I have the enthusiasm to bridge the gap between technical concepts and ordinary people, and I care deeply about finding the best possible solution for whatever problem people are having.  

