\documentclass[12pt]{scrartcl}
\title{Can a paper be autopoietic?}
\subtitle{And other metaphysical questions}
\author{Ryan Spangler}
\date{\today}

\usepackage{commath}
\usepackage{graphicx}
\usepackage{listings}
\usepackage{amsfonts}
%% \usepackage{fontspec}

% python highlighting ----------
\usepackage{color}
\usepackage{listings}
\usepackage{textcomp}
\usepackage{setspace}
\usepackage{palatino}

%% \doublespacing
\usepackage{xcolor}
\usepackage{titlesec}
\setkomafont{title}{\normalfont\huge}
\setkomafont{section}{\normalfont\huge}

\setcounter{secnumdepth}{0}
\linespread{1.5}

\begin{document}
\maketitle

\section{Genesis}

Can a paper, such as this one, composed by a human, symbol by symbol, word by word, sentence by sentence, paragraph by paragraph, even section by section until ultimately there is a whole, a unit, a pod of thought full of cross-referencing ideas and mutually reinforcing statements, can a paper such as this ever be truly and fully autopoietic?  Why or why not?  What is necessary for something to be considered truly autopoietic?  What kind of essential relationships are needed between the parts of an autopoietic system to realize this sweet and ineffable quality we call autopoiesis?  These are the questions we will explore in this paper.  

To begin, we will define our core term and idea: autopoiesis.  Autopoiesis is the process of an entity creating itself.  Because of the inherent self-reference in this idea, the causality spins around in a loop and can be unfolded indefinitely, as in the notion that an autopoietic entity is a process that gives rise to and sustains the process that is giving rise to that process... ad infinitum.  The process is itself playing (at least) two roles, that of the generator and the thing being generated.  It is the product of that process of generation.  As the generator, it converts arbitrary form into the specific form of its being, and as a product it is this form which is generated, the form of the generator of that form.  This is a very special relationship between the components of an autopoietic system and one we will explore throughout this paper.  

This topic is fraught with uncertainty.  Though the general idea is clear, it is still dimly understood just what is necessary to call a system autopoietic, or how one would go about constructing an autopoietic system, or even what kind of dynamics are involved or required.  Thus this paper will be exploratory in nature, a process of discovery and generation itself.  I am hoping through the process of writing this paper I will be able to apply the notions of autopoiesis not just conceptually, but to build it directly into the structure of the paper itself.  Whether the answer to our question posed in the title is positive or negative, in as much as a paper can be autopoietic this one will be.  For this reason there will be much discourse of a self-referential nature as I tie the subject of the paper back to itself in various ways.  

Also, as the subject is inherently non-linear, so is the structure of this paper.  There is no sequence here, everything relates to everything else.  As much as I am able to tie any part back to any or all of the other parts of this paper I will do so, either explicitly or implicitly (which will be most of the time the latter).  It is an unfortunate but necessary challenge that this paper is nevertheless composed at its most basic level of a sequence of symbols.  The tension between the nonlinearity of the subject and the inherent linearity of the basic structure of the paper will be a subject itself in the section ``Turning linearity into full interconnectivity'' (notice this previous statement itself embodies a link between different parts of the paper, already forging connections between this essentially one dimensional medium, as well as providing an implicit connection through example to the original topic of this paragraph.  I will hopefully not continue to discourse too much on the specific occurrence of these links (though I will discourse on the nature of these links in due time).  This was merely an episode for demonstration's sake.  Though I do feel it somehow necessary later on to refer to my own discourse about my discourse, since it is the thing that this whole paper is composed of on a level somewhere above that of symbols).  

Already in the ``Genesis'' the seeds of autopoiesis have been sown.  It begins with a description of how to construct the paper itself, ``symbol by symbol, word by word, ...'' \cite{Spangler}.  From this basic tenet the rest of the paper unfolds.  Whatever thoughts are attempted to be conveyed, the process will be one of symbol catenation, symbol removal, the splicing of large blocks of symbols, etc.  They are all essentially operations involving blocks of symbols and individual symbols.  Could this description of the process of paper creation itself ever \emph{become} the process of paper creation?  How does a description of a process become a process?  This is at the heart of the notion of autopoiesis.

Which brings me to my next foundational topic.

\section{The structure of this paper}

The structure could be considered to be many things.  I will run through a number of them, and then attempt to explain where they all come from and what could be the nature of this thing we call ``structure''.  

First off, as I have been saying all along, the inherent structure of the paper is the string of symbols that compose it.  We have already assumed there are levels above this, but are there any levels below this one-dimensional crystal of the sequence of alphabetic symbols?  Of course, the notion that the paper is composed of symbols says nothing about the \emph{specific} symbols that will be produced.  Seemingly, they could be typed out by hand, printed ink from a printer or even the pattern of rendered pixels on the screen.  The specific medium is unimportant (or at least not essential).  So really it could be considered to be composed of bits, or raw information.  As I am using a program to type this, we could even consider it to be the specific bits stored on disk that represent the file that ultimately gets printed in some human readable form.  This translation from raw bits to readable form is not necessarily obvious just looking at the bits, but requires a specific program instantiated in a specific way, which itself is simply a glob of bits stored somewhere in a computer.  Bits generating other bits.  Could those bits generate themselves?  But what is doing all this generating inside the computer?  The bits reside on disk, but the changer of the bits is really an electronic process informed by but of a different nature than the bits.  We will explore the possibility of a computational process generating itself in the section ``Computational autopoiesis?''  Suffice it to say, there are many ways we could look at the notion of the ``structure'' of this paper.

The point of all this is that there is a specific level of granularity which really captures what the nature of something really is.  What is truly the structure of this paper?  Going above the level of symbols, is it still the same paper if translated into French, or Chinese?  This reveals that the paper is actually composed of thought stuff, of which the symbols are merely a shadow of, a projection of thought into the realm of marks and bits.  The symbols capture the ideas in some way that can be transmitted long after I have finished composing them, but they are not \emph{the ideas themselves}.  The ideas go far beyond mere symbols, and are living things which reside in the minds of those who have consumed them.  

So the notion of ``structure'' really depends on the notion of ``level'', and we can't talk about the structure of anything without also making clear what level of the subject we are talking about as well.  In this way structure could mean the totality of all of these levels superimposed over one another.  Structure is a multi-scale concept.  We can talk about the structure of the words, their composition and juxtaposition.  That would be a level higher than symbols but below ideas.  The structure we really care about in this paper is the structure of the ideas relative to one another.  This makes the whole notion of an autopoietic paper seem much more plausible, as it is not necessary to have the paper truly generate the substance of paper upon which it resides, but rather it is an idea which generates and is generated by the idea itself.  The ``paper'', in terms of the physical medium of transmission of these ideas, the fibers holding the specific configurations of ink, is merely a projection of this true, organic, conceptual structure of the ideas contained within it.

Before we explore that further, let's truly entertain the notion of what could constitute a paper which fully gives birth to its own substance and form, even though ultimately this is not the form of autopoiesis we will achieve in the current paper.

\section{Thought experiment: A physically autopoietic paper}

What is the difference between paper and the thing that makes paper?  How could these roles be merged?  

Imagine a sheet of paper, clean white.  Real paper is composed of fibers, which were once wood.  For this experiment let's imagine these fibers don't necessarily need any link to anything that was once alive.  They can be idealized fibers of homogenous material.  In order to be autopoietic, these fibers cannot be static, but must somehow be involved in the process of generating these fibers.  Possibly each fiber is a fiber creation machine in its own right, or at least a fiber-renewing machine.  What is the duty of these fibers if they are to create an autopoietic whole?  

A regular sheet of paper persists because it is static, but when faced by things like creasing and tearing it is mute in response.  Let's consider these events to be the first kind of cataclysmic injuries that would require an autopoietic response in order to perpetuate in the face of.  So for now we can say that an autopoietic sheet of paper would resemble an ordinary one if no trauma were to befall it.  If paper is creased, that means that a series of fibers or the connections between fibers have been broken.  For clarity only we will assume that the fibers are atomic, and cannot themselves be divided, so that the creasing of a sheet of paper involves the selective severing of certain connections between fibers, effectively weakening the sheet along a certain axis.  In this light, tearing is just an extreme form of creasing where the connections between two subsets of the fibers of a single sheet become entirely disjoint, so that if we solve the problem of creasing, we also solve the problem of tearing as an extreme case.  To start off we could endow the fibers with the ability to fuse to another fiber, possibly only at a terminal active site and at a particular orientation (to preserve the sheet-like quality of the aggregate).  So any unfused fiber terminal would be continuously seeking a complementary unmatched fiber with which to fuse, leading to an aggregate behavior of repairing creases or tears.  

Note that tearing and creasing are problems in the context of the definition of the piece of paper as a flat and continuous whole.  There is something suspiciously circular about this idea that problems are things which deviate from the true form.  Isn't also the true form the true form because it is the thing which other forms tend towards?  As a kind of attractor, the autopoietic form calls itself into being because that is the form that it generates, and it steers deviations back into the autopoietic path.  If it did not do so, it would dissolve and therefore never become autopoietic.  Once again we see the ambiguity between the product of the process and the substance, or dynamics of the process.  Both are one and the same.  The blurring of these distinctions is forced on us, so perhaps this distinction was illusory in the first place?  There is an amazing symmetry to autopoiesis in that many things which are separate in many other cases become superimposed and unified when autopoietic, like blurry double vision converging into a single image.  Perhaps autopoiesis is the fundamental state, and other things are deviations from this original unity?

So now we have a piece of paper that when creased, straightens back to its former shape.  When torn yet brought back together, the paper will magically fuse back into a seamless sheet, but if the severed halves are lying apart from one another, they won't really have the chance to fuse.  This is dissatisfying, and can be rectified in a number of ways.  The first is to imbue the fibers with a kind of metabolism which breaks down various elements in their environment and grows other fibers from them.  So if we introduce the notion of an ``incomplete'' fiber, such an incompleteness drives the fiber to dissolve whatever it is touching and integrate it into restoring the fiber to its full length.  If a fiber was complete yet not attached, instead of waiting for another unattached fiber as before, it will start the seed of a new fiber, the first circle of unfulfilled existence that subsequently catalyzes its own growth.  This kind of active digesting and replicating fiber would have the effect of turning the sheet of paper into a kind of endlessly growing surface of paper, rendering all nearby material into part of the paper form.  If the paper were ever torn, it would just grow back from the new edges created by the tear.  If you ever happened to touch it, it would cleave through your flesh as the surface consumed a plane of material through your body.  

Obviously this is dangerous and undesirable, but nevertheless demonstrates the idea.  This is getting closer to our goal of a fully autopoietic sheet of paper, but part of why this still remains unsatisfactorily not-yet-autopoietic is that it grows unboundedly.  Part of the notion of autopoiesis is that it maintains a kind of identity in the face of constant change.  How could the sheet remain a sheet, but repair itself otherwise?  This is the kind of thing that may require global coordination of the elements to some degree, so that the fibers out at the end of the sheet would have some kind of signal that they were not to grow anymore, that would still allow a circle cut into the center of the sheet to be filled in through metabolism.  This ventures into the realm of biological development and the concept of neoteny: the triggering of the cease of growth and the relative timings of various mutually growing systems.  Neoteny is a description of the thing that leads fingers to stop growing at a certain point, for example.  We can see our innocent sheet of paper is becoming more and more creature-like, and this is revealing.  

Another approach is to have the sheet grow forever, but also have it curve slightly inward, so that it ultimately joins up with itself along the edge.  This obviates the eternally expanding concern and provides us with a surprising link between our autopoietic sheet of paper and the cell: the ultimate inevitability of the sphere as a reasonable means of turning infinity into a finitude.

Just by trying to make something like a sheet of paper autopoietic, we cannot help but imbue it with seemingly biological properties.  Biology can be seen then as the study of autopoietic entities and their relationships to one another.  

However, this says nothing about whatever is written on that sheet of paper.  Nothing about the message contained in the paper contributes to the physical structure of the paper itself.  Paper is functionally identical whatever happens to be written on it.  How could this causal loop be completed?  How could the nature of the message itself inform the structure of the paper?  Perhaps in some miracle of linguistic origami?  Or sculptural alphabetics?  If the message written on the paper were able to influence the way that paper moved or repaired itself, it could be considered autopoietic.  And again our paper is becoming uncannily life-like.

\section{Biology and autopoiesis}

Is there anything biological that is not autopoietic?  Is there anything autopoietic that is not biological?  

Let's take a look at how biology goes about being autopoietic, as a kind of canonical example of the genre.  We can take one of the simplest life forms known to humankind, the Escherichia coli bacterium, as possibly the purest and simplest example of a living, autopoietic system.   \cite{Alon}

At its most basic, the bacterium is a membrane of phospholipids with an assortment of molecules jammed inside.  The processes inside the membrane first and foremost maintain the integrity and health of that membrane.  This is something that is essential to Maturana and Varela's conception of autopoiesis \cite{Maturana}, that the autopoietic process itself makes a clear spatial distinction between what is inside it and what is not, what is \emph{part} and what is not.  There are a variety of roles the subsystems inside of the membrane play, and there are also many selective pores embedded in the membrane providing an interface to the outside world.  It is a hub of activity.  

One kind of role already mentioned are the ones that restore and repair the sheet/sphere of lipids that comprise the membrane.  These depend on a kind of energetic currency, along with many other processes in the cell, ATP.  One of the main engines of cellular events, the humble ATP (adenosine triphosphate) is involved in nearly everything that happens in the cell.  The main reason for this unique role is that it provides a large burst of energy for very little energy input: the extra third phosphate is poised to break free, showering upon its catalysts the energy to power their own motive force.  This energetic release is coupled to innumerable cellular processes.  So here we have something like a first role, that of energetic storage and transmission, played by ATP.  

Maturana and Varela speak of the processes necessary for autopoiesis:  ``An autopoietic system is organized as a bounded network of processes of production, transformation and destruction of components that produces the components which: (1) through their interactions and transformations continuously regenerate and realize the network of processes (relations) that produced them;  and (2) constitue it (the machine) as a concrete entity in the space in which they (the components) exist by specifying the topological domain of its realization as such a network.'' \cite{Maturana}

So how do cells embody the production, transformation and destruction of these elements?  There are enzymes that catalyze the production of pre-membrane components, powered by ATP (production).  Then these agents are transformed by another series of enzyme/catalysts into something that can actually be integrated into the membrane.  Then still other enzymes finally insert these new elements into the membrane.  During the lifetime of a membrane, multiple agents continuously work on the fabric of the membrane, inserting and pulling out membrane spanning pores, sensors and effectors, and repairing the base level lipid structures that form the membrane itself.  In return, the membrane shelters and coheres the activities of the cell.  The membrane also acts as a battery, with selective pores that push out or pull in certain ions, but not the opposite.  This creates a charge differential on the two sides of the membrane which is then further harnessed to do work in the cell.  

Already we have a circular relationship, one of the prerequisites of an autopoietic system.  The membrane depends on the structure creating and maintaining enzymes that reside in the cell, but also makes possible the processes that give rise to these agents in the first place.  The conditions for their existence depend on a membrane that they maintain, and that would not exist with out them.  This mutually causal circularity is at the heart of any autopoietic system.  

This is not the only circularity in the cell however.  There are a number of causal cycles, all of which fuse to form a continuous network of mutual causality.  In a way, this could be thought of as a kind of causal tension/integrity, or tensegrity, as R. Buckminster Fuller conceived many years ago.  Tensegrity is a perfect description of the kind of responsive and adaptive balance achieved by autopoietic elements.  When the elements are all pulling on each other to the point where they have formed a stable balance between one another, any disruption is propagated and absorbed by the whole.  This is where Maturana and Varela made a distinction between structure and organization, where in an autopoietic system structure is constantly changing but organization remains constant.  The organization comprises a self-referential tensegrity between the various components which is constantly adapting its structure to maintain.  

What is the nature of this organization?  It is a network of causal links, which is an important concept and one which is fundamental to our pursuit.  We will get to this in the section ``Causality and autopoiesis''.

One other conspicuous feature stands out about the simple bacterium we described earlier.  There is also, spanning the bulk of its volume, a \emph{sequence} of characters in a four letter alphabet, which forms another link in this mutual causality that is the cell.  This sequence is being continually read and duplicated, which a great degree of the resources of the cell is invested in its maintenance and propagation.  This is the fabled DNA, the code of the cell.  However, as it is sometimes deified into an absolutist kind of controller position over the activities of the cell, we see a quite different story when looking more closely.\cite{Ho}   It is largely a passive entity, with other actors playing a much more assertive role in the day to day activities in the cell.  There are busy molecules that bind to this leviathan, split its helix and transcribe short snippets, while these get hurried off to the ribosomes which do the work of actually assembling the amino acid chain that will become the protein in question.   The protein product of this process is now an active member of the molecular community, going about its business, contributing its small gift to the fabric of activity that gives rise to the cell as a whole.  Compared to these frenetic go-getters, the DNA molecule just sits there, waiting to be read and duplicated.  It is a great repository, but it is no actor.  The actors refer to it, depend on it for their sequences and in this way the DNA monster could be considered to \emph{encode} the protein, but it is far from actually performing any of the countless chemical and energetic tasks that are required to keep the cell whole.  What is this paradox of the least active yet most important element?  

Clearly, the various proteins would be an uncoordinated jumble if they were not encoded for by the DNA.  How would the forms that give rise to these proteins be expressed, if there weren't some universal informational medium present in the cell to encode for them?  It is a question as to whether this role played by the DNA, this genetic code, is really an essential one to autopoiesis, or if it is a greater feature of the subset of autopoietic systems known as life.  Is this feature of life as we know it and subsequently our best examples of autopoietic systems a necessary condition for that system to be autopoietic?  This is an important question which I will explore further in the next section.

\section{Is a genetic code necessary to autopoietic systems?}

The genetic code is popularly considered to be the executive of life, the brain of the cell.  This is wrong for many reasons, and a classic example of trying to impose human hierarchy onto a fundamentally interconnected and inseparable situation.  It plays a role and has a relationship to the rest of the cell's activities, but it is barely an actor at all, as previously demonstrated.  It is a repository, a library of information that is used to construct various cellular components.  It is a book of recipes for molecular machines, but it is only because of the properties of these molecular machines that this book has any meaning.  Otherwise it would be a long string of unrelated symbols.  In this way we draw a parallel between the paper we tried to imbue with autopoiesis before and the cell and its genetic material: each has a sequence of symbols embedded in it somehow.  Is this metaphor possibly further explorable? 

To go back to our thought experiment, imagine that the sequence of characters in the paper was read by fibers in the paper.  For this to be the case there may need to be fibers that flow freely over the surface of the paper, tracing the outline of the symbols and translating this motion into the form of a new fiber.  This process would create a new fiber that had particular qualities - perhaps it acts as a signal fiber to the extremities as a means to prevent them from growing at the edges of the sheet, or a sensor fiber that detected different degrees of pressure or variability, or perhaps they were a regenerative fiber awakened from dormancy to oppose a tear somewhere in its fabric.  So certain conditions awakened certain portions of the sequence of symbols cast on the sheet, triggered a process that translated that sequence into the signature in a fiber, sending that fiber off to perform its duty in the larger surface of the paper.  This is one way to connect what was previously unconnected, the symbols on the paper from the maintenance and regneration of that paper.  In certain ways, the paper and the cell are becoming more alike.  One key progression in this most recent elaboration of the paper thought experiment is the differentiation of the fibers role and capabilities within the sheet.  The fibers are constantly interacting, but they are not identical.  

So in our picture of autopoiesis now we have identified some main elements:  a membrane separating inside from outside, the varied agents of activity within the membrane, and a large mostly static repository of information whose sequence encodes the relative varieties of active agents within the system.  These different roles are highly interconnected in terms of how events unfold, and those relations take the form of a \emph{causal} network.  In the next section we will explore what that means in detail.  

So is this role that DNA plays an essential one for autopoiesis?  The accretion and accumulation of the information necessary to continue its existence is most likely an inevitable result of an autopoietic process.  Beyond being necessary for autopoiesis to exist, it may be also that autopoiesis by nature creates a repository of information in this way.  The process of evolution would reward the system that does not need to rediscover every molecular configuration in order to perpetuate, so the successful cells will aggregate more and more of this knowledge as time goes on.  Of course, there is a cost in each base that is part of this master sequence, so the accumulation of information is bounded to plausibility, and there is some pressure to make this sequence the most efficient one possible.  It is plausible that a fledgling autopoietic system may not start out with a genetic code, but would quickly acquire one simply as a byproduct of its natural behaviors.  

\section{Causality and autopoiesis}

Causality is an elusive yet potent concept, and lies at the heart of everything else that occurs in our universe.  We could not witness anything if it weren't for the causal relationship between that event and some portion of ourselves sensitive to such events.  Our ``senses'' are really means for causal informational events to register in our network of transformative processes, and our effectors are those events originating from within us that impact the external world in some causal way.  From a high level perspective, every relationship is a causal one, and if something has no causal relationship to anything we experience, it may as well not even exist.  

Still, causality is a suspicious concept, in that in spite of the fact that it seems to be behind every physical event we witness, it does not itself seem to have any physical qualities.  Causality has been one of the most discussed and controversial of philosophical concepts, from the time of Aristotle and his ``four causes'' to present day right now in this paper.  Is causality actually a thing, or is it a description of how we witness related events?  Wrapped up in causality is the notion of time, with events that cause other events coming before them in time.  What is this causality, and can we put a more precise meaning to this concept which seems to be at the very center of our quest?

Rosen was preoccupied with the notion of causality in his book ``Life Itself''.  \cite{Rosen}  Rosen was also searching for some way to capture the notion of autopoiesis as a formal system.  Citing the limitation of Newtonian thinking and modeling as the presence of only recursive causality (where the next time step depends entirely on the previous timestep and an algorithm to transform that timestep into the next timestep), he attempted to expand what kinds of causality are brought to bear on the problem of autopoiesis and cyclical self-generation.  Instead of time steps being endlessly transformed to subsequent time steps through a static and unchanging process of the ``Laws of Nature'', Rosen mused, what if the process itself was the product of something else?  This process of transformation was given a suggestive name, Metabolism.  However, rather than simply metabolize A into B, as Newtonian models do, how about if metabolism (M) itself were generated by something?  In general, Rosen was trying to ensure that everything in the system was entailed by something else.  

Rosen gave A the role of the Environment, the arbitrary forms that get transformed into B, Behavior as he calls it, the specific form, by M, the Metabolism.  So metabolism creates behavior from the environment.  Behavior itself is a process that takes M as its raw material and produces Repair, another mapping which consumes the structure of Behavior and produces M again.  And with that the cycle is complete, cycling endlessly round and round.  For notation, we will say $M:A \rightarrow B$ to mean M turns A into B, or M generates B from A.  These relationships can then be written

$$ M:A \rightarrow B $$
$$ B:M \rightarrow R $$
$$ R:B \rightarrow M $$

In this way, each element is also a process and a product.  Each element consumes something and produces one of the other elements.  This is Rosen's equation, and an admirable first attempt at the formalization of autopoiesis.  

There are a couple of notable qualities to Rosen's equation.  First is that the types of the domains and ranges of these mappings are not really finitely nameable in that they are circularly dependent, so that a full exposition of the type of each equation would ultimately involve its own definition again.  This is an interesting occurrence of circularity which finds its way even to an attempted formalization of the living process, and further evidence of the inherent role its own circular definition plays in the construction of an autopoietic system.  

In this light, we have a potential mapping of our previous discussion of cellular autopoiesis to this system Rosen has outlined.  Metabolism is clearly the activity of the various molecular inhabitants of the cell, transforming raw materials into more cellular entities through their endless cycles of structural tweaks and modifications.  The product of metabolism, which is the structure of the cell, could be considered Rosen's Behavior.  If so, in the Rosen equation Behavior is an entity which consumes the metabolic agents and produces Repair.  Repair itself consumes behavior and produces the processes which drive Metabolism.  This circularity is boggling, but some sense can be made.  If Repair produces the units of metabolism, then repair could be considered the process of gene transcription and translation that produces the proteins involved in metabolism.  In this scenario the genetic code is a \emph{part} of repair, but not the sole element of repair.  The agents of gene regulation and transcription work together to produce the role of repair, or replacement and maintenance of the balances of molecular entities throughout the cytoplasm.  

Where Behavior is the product of Metabolism, it is also its consumer (to produce Repair).  So metabolism creates its own consumer in a kind of apocalyptic relationship between these roles.  But Repair itself is the thing which restores Metabolism, so the story has a happy ending after all.  Well really, it doesn't end, that is the point.  These roles and activities chase each other around in an endless elaboration of the classic ouroboros.  Where that sad creature just devoured itself, this beast devours one part while giving birth to the other part which devours the original, giving birth to the perpetuator of everything else!  What strange mythology Rosen makes.  

So for the cell we could consider Rosen's Behavior to be the membrane and its enclosing, difference creating purpose.  Zwick points to this idea of creating a difference as the first and most significant act of existence.  \cite{Zwick}  Creation is the act of forging a distinction between something and something else, because before that, everything was indistinguishable.  This also relates back to the notion of tensegrity, with the creation of difference resulting in a tension between the opposed differing elements.  In this way too, the creation of positive and negative charge could be considered this first kind of distinction, where a force is created directing this difference to be extinguished, and thereby driving the cosmos forward and orienting this immense arrow of time.  

So we have metabolism generating behavior, behavior being the thing that generates repair, where repair is generating the metabolism in the first place.  If metabolism only exists, it will begin to generate behavior, which then consumes that original metabolism in a temporarily cannabilistic act before repair starts creating metabolism again to replace what the behavior was taking.  So metabolism alone, if structured correctly, could give rise to all the others.  

This suggests that life as we know it could have begun as a metabolic process (M), which only later as a byproduct generated a membrane as a result (B), which set the stage for the accumulation of information about the form of those metabolic elements in the structure of the repair (R).  And once the cycle was completed the entire system just kind of took off, and we know it to this day as the great tree of life of which we are all part.  

So as a key feature of all of this lies the multiple kinds of causality present between the elements in this system.  Each element is a material cause, but also an efficient cause.  And in being the product of something else is also the final cause of the entire chain of relationships.  This Rosen expounds on to some length in his book, but we can take away that his conception of a living, autopoietic system is one that is maximally causally related, which is in tandem with our own thoughts along this philosophical journey.

\section{Computational Autopoiesis?}

Can a program be autopoietic?  Rosen claims resoundingly No, that life because of its circularity cannot be simulated, but must be always instantiated directly in a physical medium, so that the various arrows of causality may all flow unhindered from element to element.  While this may be true in principle, this reminds me of claims that certain transcendental numbers like $\pi$ can never be fully calculated, but that they are doomed to be only perpetually more finely approximated.  This may be a metaphysical tragedy, and linked to all other tragedies of the finite, as Zwick would remind us,\cite{Zwick} but in practice we use $\pi$ all the time, and I would expect no less from a simulated, computational model of autopoiesis.  While a real man-made instantiation of autopoiesis may require robotics to achieve in order to finally fuse the physical in a circle of causality, I see no reason why autopoiesis can't be usefully simulated in a computational setting.  The question is what form this would take, and how it could be related to the rest of computation in a useful way.  Perhaps an autopoietic program could find balance between different forces present in the environment (informed through sensors and acted out through various programmatic controllers of physical devices that ultimately trigger changes in the signals received by the sensors), or an autopoietic programmatic system could be used to accumulate a genetic code of useful forms in an evolutionary setting, with the intent to later propagate this accreted knowledge to other agents in the field.  Possibly an autopoietic program could be used to create computational membranes between different effective systems in the larger computational context, and through these membranes and as the systems contained within them struggled to maintain their autopoiesis, the various success or failure of different discovered sequences could inform the nature of the process these systems are reacting against.  

To start on such an endeavor would require the identification of the necessary components required to sustain autopoiesis, and insight into the relationships that define the network of causality between them. A process which constructs a membrane to contain the elements constructing that membrane would be a good start towards an autopoietic computation.  If there could be a kind of space where functions could consume other functions and give rise to still other functions, all of which ultimately are able to consume any arbitrary stream of function/structure, then these functional units could be oriented towards one another in a way that created cyclic causality.  Is this enough in itself to create autopoiesis, magically from relationship alone?  I do not know, but I suspect not.  However, in due time I will do these experiments, and more, in an effort to really create and discern the possibility of a (n approximated!) computational autopoiesis.

\section{Turning linearity into full interconnectivity}

In what way could this paper be considered autopoietic?  Any at all?  Is the entire endeavor as ridiculous as it originally sounds?

There is a kind of impossible quality about this undertaking.  Not just impossible in that it is still dimly understood how exactly to construct an autopoietic system and what dynamics are actually necessary in that process, but also in that in a real way, a physical paper is too inert to constitute an autopoietic unity.  This linear sequence of symbols can only refer to itself by appealing to a higher level of structure which is composed of those symbols, but also riding above them like clouds above a horizon.  In this way the particular gesture that the flatness of the linear crystal of symbols makes as it unfolds into a mass of interconnections, a network of meanings, is the genetic identity of the real substance of the paper.  The paper is not just the symbols, though the symbols are necessary for its transmission, but it is also the thoughts that it triggers in the minds of those that encounter it.  This is the analogue of the metabolism and the membrane in the cell.  In this case, ideas live in the mind, the mind plays host to these entities that are generated from the genetic code of words in sequence, but whose bodies occupy the mental realm just as genuinely as cells inhabit the realm of space and matter/energy.  The words trigger the creation of mental agents, structures who are reactive with the other mental structures that inhabit the mental landscape (perhaps you can feel them taking shape right now!).  The other remarkable thing about these mental structures is that they are composite entities.  A single idea will exist distributed across a whole host of minds, its body dipping into each, somewhat submerged among all the other ideas, not totally in one mind or another.  Ideas transcend individual minds, though depend on them, much in the same way cells depend on yet transcend the individual molecules that compose them.  An idea defines itself this way because its boundary is also distributed, so under Varela's definition defines inside from outside cutting across the boundaries of many minds.  It is a curious thing when we think of ourselves as indivisible to know we are merely playing a part in hosting a variety of occupying and interacting idea-fragments, all of which are beams from the prism of everyone's collective mind splitting the signal from its particular originating informational structure.  

Thus turning linearity into full interconnectivity is very much what the cell is doing when it translates genetic sequences into the proteins those sequences encode.  A sequence generates active, interacting agents.  So a key to this whole situation is a universal system for converting arbitrary sequences of a particular medium into elements in whatever system the autopoiesis is adapted to.  This is why DNA is necessary to the cell, and something like it could be necessary to all autopoietic systems: there must be some way to specify arbitrarily complex structure.  The $DNA \rightarrow protein$ channel is a beautifully crafted example of an infinitely expressive system for specifying and creating any arbitrary molecular configuration, which can have any imaginable chemical/energetic effect on other products of this process.  It is the perfect setup for generating any structure.  This is why the $DNA \rightarrow protein$ system is so powerful.  It can truly make anything.  

Linearity into full interconnectivity, much like the whole process of writing and reading this paper!  Now we can finally turn to the original question at hand and achieve a kind of operational closure before we pinch off this membrane into a continuous sphere and let it float freely along its merry way.  

\section{Operational closure}

Now we get to the crux of the paper, the final arrow that sweeps back around to include its own origin.  Can something like this paper ever be considered autopoietic?  

Well obviously, as it stands, it is not physically autopoietic.  In order to be so, it would have to essentially transform into a cell, as we have thoroughly and colorfully demonstrated.  But in another sense, it is the way in which this paper is really the ideas it contains rather than its physical instantiation as a paper that matters.  

How could the ideas contained in the paper be autopoietic?  If the ideas themselves are a process that generates these ideas, we may be able to call them autopoietic.  The ideas are the product, and also the producer, of themselves.  Are there any ideas like this?  I propose there is, and it is precisely the idea that has been under exposition this entire time!  The question of whether something is autopoietic or not, and what precisely could make something autopoietic, itself (through my assistance, as a thread of myself is part of the fabric of this particular autopoiesis, but also latent in the structure of the question itself) generates the entire rest of these ideas, and these ideas do service to perpetuate that question, whether or not something is autopoietic.  So it is not the answer to the question, but rather \emph{the question itself} which is endlessly generating itself by virtue of existing and asking the kind of thing which would perpetuate and produce its own existence.  The question is never meant to be answered, indeed an answer would mean death in a real sense, in that autopoiesis would cease.  The asking of the question, through its questioning, generates that question.  The question gives rise to the paper, and the paper generates the question, raising it from the sea like a giant pearl of self-creating majesty.

Can a paper be autopoietic?  This paper is both the seed of the answer that will be its own demise, and the perpetuation of that question, for all time.  

\bibliographystyle{plain}
\bibliography{autopoiesis}

\end{document}  

