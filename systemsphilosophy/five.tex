\documentclass[12pt]{scrartcl}
\title{Systems Philosophy Five}
\author{Ryan Spangler}
\date{\today}

\usepackage{commath}
\usepackage{graphicx}
\usepackage{listings}
\usepackage{amsfonts}
%% \usepackage{fontspec}

% python highlighting ----------
\usepackage{color}
\usepackage{listings}
\usepackage{textcomp}
\usepackage{setspace}
\usepackage{palatino}

%% \doublespacing
\usepackage{xcolor}
\usepackage{titlesec}
\setkomafont{title}{\normalfont\huge}
\setkomafont{section}{\normalfont\huge}

\setcounter{secnumdepth}{0}
\linespread{1.5}

\begin{document}
\maketitle

\section{Synchronics and spirituality}

I see a possible connection between synchronics and spirituality in the idea of Distinction as the first act of creation.  Any being is necessarily in distinction from its environment, and this tension between the distinction it is making by its very existence and the eternal wholeness from whence it sprang is a defining element of its life.  Every being fears the extinguishment of that distinction but also the universe pulls at it so strongly, and ultimately, inevitably, prevails.  Every distinction is at opposition with the essential wholeness of all things, so finds its power in its resistance of this relentless force.  Tension is at the heart of our being, and gives rise to the special tautness of experience.  The tautness of a surface receptive to any stimulus, of which the most minor of disturbances would send into a quavering ripple of excitement that spread throughout its being, absorbed and experienced and somehow integrated into the organization without changing the essence of that organization, the essence of an organization able to become any form.  The ``spirit'' is the dynamics of this tension as they play across the surface of our experience through that surface's connection to all other things.  This explains the ineffable nature and almost cosmic connection that defines spirituality.  

\end{document}  

