\title{Systems Philosophy - Two}
\author{Ryan Spangler}
\date{\today}

\documentclass[11pt]{article}

\usepackage{commath}
\usepackage{graphicx}
\usepackage{listings}
\usepackage{amsfonts}

% python highlighting ----------
\usepackage{color}
\usepackage{listings}
\usepackage{textcomp}
\usepackage{setspace}
%\usepackage{palatino}

%\doublespacing

\setcounter{secnumdepth}{0}

\begin{document}
\maketitle

\section{Tentative Paper Topic}

I chose to take this course instead of writing a thesis, so I am taking this opportunity to write a mini-thesis on what was going to be my thesis:  autopoiesis.  I care about autopoiesis because I believe in order to understand life we will ultimately have to understand autopoiesis.  Also, it is the prime example of a kind of relationship between a totality and its parts that exemplifies the phenomena that systems science was developed to tackle.  Autopoiesis cannot be understood simply by examining the parts themselves, it is an essential relationship between these parts that creates something beyond any of the parts themselves, a living whole that is self-renewing and self-perpetuating.  To me this is the essential systems relationship, a whole whose parts could not exist without the whole, and whose whole cannot exist without the parts.  This is all related to systems philosophy and the relations between parts and whole, structure and function.  An autopoietic entity is the perfect marriage of structure and function, wherein the function is to maintain the structure, and the structure generates that function.  It is self-referentially defined formation of entities.  

I feel that a real understanding and model of autopoiesis could contribute greatly to the systems research program in that it would be a real example of something that fundamentally cannot be approached by traditional reductionistic means.  It is an important topic that can only be dealt with through systems thinking and systems modeling.  A theory of autopoiesis could be something the systems field points to as a great success analogous to how Newton's laws of gravitation revolutionized physics and ``natural philosophy''.  Possibly waxing too grandiose here, but it is not a totally unrealistic claim.  

\end{document}  

