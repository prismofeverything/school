%%%%%%%%%%%%%%%%%%%%%%%%%%%%%%%%%%%%%%%%%
% Plain Cover Letter
% LaTeX Template
% Version 1.0 (28/5/13)
%
% This template has been downloaded from:
% http://www.LaTeXTemplates.com
%
% Original author:
% Rensselaer Polytechnic Institute 
% http://www.rpi.edu/dept/arc/training/latex/resumes/
%
% License:
% CC BY-NC-SA 3.0 (http://creativecommons.org/licenses/by-nc-sa/3.0/)
%
%%%%%%%%%%%%%%%%%%%%%%%%%%%%%%%%%%%%%%%%%

%----------------------------------------------------------------------------------------
%	PACKAGES AND OTHER DOCUMENT CONFIGURATIONS
%----------------------------------------------------------------------------------------

\documentclass[11pt]{letter} % Default font size of the document, change to 10pt to fit more text

\usepackage{newcent} % Default font is the New Century Schoolbook PostScript font 
%\usepackage{helvet} % Uncomment this (while commenting the above line) to use the Helvetica font

% Margins
\topmargin=-1in % Moves the top of the document 1 inch above the default
\textheight=8.5in % Total height of the text on the page before text goes on to the next page, this can be increased in a longer letter
\oddsidemargin=-10pt % Position of the left margin, can be negative or positive if you want more or less room
\textwidth=6.5in % Total width of the text, increase this if the left margin was decreased and vice-versa

\let\raggedleft\raggedright % Pushes the date (at the top) to the left, comment this line to have the date on the right

\begin{document}

%----------------------------------------------------------------------------------------
%	ADDRESSEE SECTION
%----------------------------------------------------------------------------------------

\begin{letter}{Markus Covert \\
Principal Investigator, Covert Lab \\
Shriram Center for Bioengineering and Chemical Engineering \\
443 Via Ortega \\
Stanford, CA 94305}

%----------------------------------------------------------------------------------------
%	YOUR NAME & ADDRESS SECTION
%----------------------------------------------------------------------------------------

\begin{center}
\large\bf Ryan Spangler \\
4235 SE 11th \\
Portland OR, 97202 \\
(503) 781-3891 \\
ryan.spangler@gmail.com

%\vspace{20pt} \hrule height 1pt % If you would like a horizontal line separating the name from the address, uncomment the line to the left of this text
\end{center} 
\vfill

\signature{Ryan Spangler} % Your name for the signature at the bottom

%----------------------------------------------------------------------------------------
%	LETTER CONTENT SECTION
%----------------------------------------------------------------------------------------

\opening{Dear Markus Covert}
 
I am writing to express interest in the Software Engineer position in your lab. Eran Agmon is an old friend from the Systems Science program at PDX where we shared Agent Based Systems, among other classes, and he reached out to tell me about it. 

I am a Software Engineer with 11 years of professional experience in a variety of fields, most recently Computational Biology at OHSU. My emphasis is on designing highly distributed and concurrent systems with a priority on performance, and graph/network processing and analysis. My highest engineering value is simplicity. In a seeming contradiction, it turns out it is easy to make something complex, and hard to make something simple. In my experience this is because a real codebase is never finished, it is an ongoing process that is continually added to, and there is no inherent reason these additions harmonize with what is already there. There must be a continuous evaluation of the system as it grows and changes, and effort put in to integrate new changes in a way that works with the existing design. It takes commitment and dedication to achieve, but with the right patterns established and adhered to, this effort becomes a delight rather than a chore.

I started coding when I was six (my dad brought home a Commodore64 one day), and I have been coding ever since. However, I have discovered that code is not an end unto itself but rather a medium for the imagination, and what has captured my imagination is the cell. I find it endlessly fascinating to consider the various ways the cell functions, the different intertwined levels it operates within and between and how such an unimaginably complex system can function so effortlessly, so \textit{tenaciously}. It is to me the ultimate mystery and challenge of our time to compose an intelligible picture of cellular function. This is why I was excited to hear about the position. Composing separate models into a coherent whole, building a system to gather all of the information we are continually discovering about the cell and sythesizing it into a large interactive model is something I would love to be a part of. 

When I first came upon your whole cell model I was astounded. Many of my colleagues here told me we weren't at the level where we could model the cell as a whole, and said that we don't know enough about the protein interactions to reliably simulate the various pathways. "Sure, but we can \textit{try}, right?" I went hunting for someone else doing work like this and found Covert Lab, and there it was! Replicating in front of me, expression flickering, pathways lighting up. It was an unforgettable experience, and in a way, just the beginning.

I know there is still a lot of work to do before we have everything figured out, and I am applying today because I want to help us get there. Thanks for taking the time to consider my application.

\closing{Sincerely yours,}


%\encl{Curriculum vitae} % List your enclosed documents here, comment this out to get rid of the "encl:"

%----------------------------------------------------------------------------------------

\end{letter}

\end{document}
