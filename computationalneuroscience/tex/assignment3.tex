\title{Computational Neurophysiology - Assignment 3}
\author{Ryan Spangler}
\date{\today}

\documentclass[12pt]{article}

\usepackage{graphicx}

\setcounter{secnumdepth}{0}

\begin{document}
\maketitle

\section{Potentiation}

Potentiation for a synapse is the adjustment in sensitivity to presynaptic activity based on postsynaptic activity.  There is biophysical evidence that potentiation is dependent not just on frequency, but on the timing of the presynaptic signal relative to the onset of a postsynaptic spike.  If a spike arrives right before the postsynaptic neuron spikes, the synapse will be strengthened.  Any other time the spike is weakened, with the depth of the depression relative to how soon after the postsynaptic spike the presynaptic spike occurs.  

If a signal is coming in at a low frequency, there is less chance for each individual spike to hit directly in the window allowing it to potentiate.  The higher the frequency of presynaptic firing, the more likely the spread will cover the area that reinforces the synaptic strength, and the more the rest of the synaptic weakenings will be offset.  This leads to high frequency long term potentiation and low frequency long term depression in synaptic strength.

\section{BCM}



\end{document} 
