\title{Computational Neurophysiology - Assignment 1}
\author{Ryan Spangler}
\date{\today}

\documentclass[12pt]{article}

\usepackage{graphicx}

\begin{document}
\maketitle

\section{Hypothesis 1}

\emph{Test the hypothesis that the action potential is an "all-or-none" event. i.e. that there is a critical level of stimulus that results in an action potential - below that level no action potential is generated, above that level no further change in the amplitude of the action potential occurs. Draw a graph of maximum membrane potential vs. stimulus amplitude.}

\section{Hypothesis 2}

\emph{Test the hypothesis that the frequency (i.e. number per unit time) of action potentials is determined by the stimulus amplitude.  Draw a graph of the frequency of action potentials vs. the amplitude of the stimulus.}

\section{Hypothesis 3}

\emph{Test the hypothesis that the density of sodium channels determines the rate of rise and amplitude of the action potential.}

\section{Hypothesis 4}

\emph{Test the hypothesis that the density of potassium channels determines the rate at which the action potential falls.  Describe the effect of reducing the Nernst reversal potential for sodium.}

\section{Hypothesis 5}

\emph{Test the hypothesis that the current through the voltage-dependent sodium channels reverses when membrane potential exceeds the Nernst reversal potential for sodium ions.  Describe the dependence of the maximal sodium conductance on the testing level of membrane potential}

\section{Hypothesis 6}

\emph{Test the hypothesis that prior depolarization inactivates voltage-gated sodium channels.}

\section{Hypothesis 7}

\emph{Test the hypothesis that the potassium channels DO NOT inactivate following prior depolarization.}

\end{document} 
