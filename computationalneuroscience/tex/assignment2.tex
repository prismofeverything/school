\title{Computational Neurophysiology - Assignment 2}
\author{Ryan Spangler}
\date{\today}

\documentclass[12pt]{article}

\usepackage{commath}
\usepackage{graphicx}

\setcounter{secnumdepth}{0}

\begin{document}
\maketitle

\section{1 - Cable Equation}

\emph{Modeling the neuronal membrane and axon as a one-dimensional cable yields the second-order partial differential equation discussed in class (with passive currents only):}

$$ \lambda^2\pd[2]{V}{x}-\tau\pd{V}{t}-V=0 $$

\vspace{6pt}

\emph{a.  Explain, with reference to appropriate membrane (transversal) and intracellular (longitudinal) resistances and capacitances, what $\lambda$ and $\tau$ are.}

\vspace{10pt}

$\lambda$ and $\tau$ are the spatial and temporal factors of the membrane potential respectively.  Each governs the order with which the space and time quantities of the membrane potential vary.  

$\lambda$ is known as the ``electrotonic length'', which incorporates the longitudinal resistance and membrane resistance of the cell into a measure of how far current spreads in the given cable.  The higher $\lambda$, the more widespread are the effects of any given current spatially throughout the cell.  Its official form is:

$$ \lambda=\sqrt{\frac{ar_m}{2r_L}} $$

Electrotonic length is in units of length, as the two resistance terms cancel out, leaving only the area term in length squared, which is subsequently rooted.  As can be seen, if the membrane resistance grows, so does the electrotonic length, since less current will be flowing through the membrane and therefore focused into the longitudinal direction.  Conversely, a rise in intracellular resistance will reduce the electrotonic length, as it takes more voltage to reach the same distance as before.

$\tau$ is the temporal order of the change in membrane potential.  It is relative to both the membrane resistance and membrane capacitance.  

$$ \tau=c_mr_m $$

This follows, as the greater the capacitance of the membrane the swifter its voltage will change.  Also, the higher the resistance of the membrane the more current will be flowing through the intracellular volume.

\vspace{10pt}

\emph{b.  What would be the effect of a greatly increased membrane resistance on propagation of the action potential along the axon? What would it mean in terms of the equation?}

\vspace{10pt}

The higher the membrane resistance, the less current is leaking out of the cell and the more that is focused towards traveling down the axon.  In the equation, both $\lambda$ and $\tau$ grow in the presence of rising membrane resistance, meaning all change in potential (temporal and spatial) is happening more quickly and over a wider area.

\vspace{10pt}

\emph{c.  i.  What is the stationary solution of the cable equation? (i.e. what happens if you set $\pd{V}{t} = 0$)?  How would you interpret this result in terms of what is happening in the neuron?}

\vspace{10pt}

If we were to set $\tau=0$ and erase all temporal activity, the cable equation becomes:

$$ \lambda^2\pd[2]{V}{x}-V(x)=0 $$

with a solution of:

$$ V(x)=e^{\frac{-x}{\lambda}} $$

In this case, only the spatial behavior (represented by x) is present.  The membrane potential is held constant, degrading exponentially along the length of the axon inversely proportional to the electrotonic length.  This would come about in the absence of any stimulus or current from an earlier neuronal section.

\vspace{10pt}

\emph{    ii.  What happens if you ignore spatial diffusion? What does this represent in terms of the biophysics of the neuron?}

\vspace{10pt}

If you ignore spatial diffusion, by setting the electrotonic length to zero for example, you have essentially a single compartment model with only passive membrane dynamics.  

$$ -\tau\pd{V}{t}-V(t)=0 $$

\vspace{4pt}

and solving with respect to t:

$$ V(t)=e^{\frac{-t}{\tau}} $$

This equation tells us the voltage will leak from the cell inversely proportional to both the membrane capacitance and resistance.  Which follows, since the higher the resistance the slower the potential will decay, and the higher the capacitance the longer it will take for the voltage to completely diffuse.

\section{2 - Transient Calcium Channels}

\emph{The cell sketched below contains T-type $Ca^{++}$ channels, $K_{Ca}$ channels, and leak channels.} 

\includegraphics[scale=0.7]{transientcalcium.png}

\emph{The conductance of the $K_{Ca}$ channels is determined by the intracellular Ca++ concentration according to the following equation:}

$$ g_{K_{Ca}}=\overline{g}_{K_{Ca}}\frac{[Ca^{++}]}{K_{Ca}+[Ca^{++}]} $$

\emph{a.  Write the differential equations necessary to model this system. Clearly define the state variables of the system. For simplicity, assume a single $[Ca^{++}]$ state, which is the average calcium concentration in the cell.}

\vspace{10pt}

For a membrane containing only transient calcium channels and calcium-activated potassium channels (along with leak-current and a point stimulus for good measure), the set of equations defining the system would be:

$$ C\od{V}{t}=I-\overline{g}_Tr^3s(V-E_T)-g_{K_{Ca}}n(V,[Ca^{++}]_i)(V-E_K)-g_L(V-E_L) $$

In this equation, the first term on the left side is the stimulus current, next is the transient calcium current, the calcium-activated potassium current, and finally the leakage current.  

In the transient calcium current there are two state variables, r and s.  r is raised to the third power and represents the onset of the calcium current.  s is the inactivation term, which counteracts the action of the channel if brought to zero.

The calcium-activated potassium current has only one state variable, n, which depends on both the voltage and the calcium concentration.  

The equation for the concentration of calcium inside the cell is:

$$ \od{[Ca^{++}]_i}{t}=\frac{1}{\tau_{Ca}}([Ca^{++}]_{\infty}-[Ca^{++}]_i) $$

This represents a continuous transport of calcium out of the cell, relative to the level at which it is currently at.  This serves to maintain a level of calcium in the intracellular space in the absence of other calcium activity.  

The other state variables have associated functions:

$$ \od{r}{t}=\alpha_r(V)(1-r)-\beta_r(V)r $$
$$ \od{s}{t}=\alpha_s(V)(1-s)-\beta_s(V)s $$
$$ \od{n}{t}=\alpha_n(V,[Ca^{++}]_i)(1-n)-\beta_n(V,[Ca^{++}]_i)n $$

You will notice that the state variable n depends on both the voltage and the calcium concentration inside the cell, whereas the other two r and s are vanilla gates depending only on membrane potential.

\vspace{10pt}

\emph{b.  Write the equations for the steady-state, or resting state, for the cell modeled in part a). Make sure you have enough equations to specify the resting values of all the state variables.}

\vspace{10pt}

The equations for steady-state are much like the dynamic equations, with all change set to zero.  Here is the full equation at steady-state:

$$ I-\overline{g}_Tr^3s(V-E_T)-g_{K_{Ca}}n(V,[Ca^{++}]_i)(V-E_K)-g_L(V-E_L)=0 $$

The equation for calcium when the concentration is not changing:

$$ \frac{1}{\tau_{Ca}}([Ca^{++}]_{\infty}-[Ca^{++}]_i)=0 $$

And the corresponding state variables are also at rest:

$$ \alpha_r(V)(1-r)-\beta_r(V)r=0 $$
$$ \alpha_s(V)(1-s)-\beta_s(V)s=0 $$
$$ \alpha_n(V,[Ca^{++}]_i)(1-n)-\beta_n(V,[Ca^{++}]_i)n=0 $$

\vspace{10pt}

\emph{c.  Add these currents to the Neuron code for the single compartment Hodgkin-Huxley model used in Assignment 1. Discuss the changes in the spiking behavior of the neuron as you increase the maximum conductances of the T-type $Ca^{++}$ currents and $K_{Ca}$ currents.}

\vspace{10pt}

In order to make sense of this problem, it was necessary to get a feeling for all of the dynamics separately at first.  This made the collective dynamics much more approachable.  In that vein, here is the dynamics of the transient calcium current alone:

\includegraphics[scale=0.8]{transientcalciumthreshhold.png}

In this case the stimulus was present for 50 ms of the total 200 ms period the trial was run, while the amplitude was raised from 0.0 to 0.03 nA over the course of all trials.  This graph is of the membrane potential of the cell.  So the first trial (darkest blue) had no current stimulus at all, yet still rose to about +20mV entirely on its own, as the influx of calcium slowly compounded upon itself, bringing the potential up.  As it rose however the calcium gate began inactivation, slowing growth until an equilibrium was reached around +20mV.  At this point there was no other mechanism to balance the level of calcium in the cell, and the potential remained constant.  Successive trials show a quicker onset of membrane potential growth, but there is a tell-tale break when the stimulus is released.  Above a certain point the gate is inactivated and no further change in potential is possible through the calcium gate alone, so the final level represents how much charge was able to be acquired by the cell's capacitance before the stimulus was released.  

Here is another view of the transient calcium current in the face of rising stimulus amplitude, this time with respect to the actual calcium current flowing.  The foremost red plot is under no stimulus at all.  As can be seen, the current rises as calcium flows into the cell, then levels off again as the cell becomes saturated with calcium and triggers calcium channel inactivation.  

\includegraphics[scale=0.8]{calciumcurrent.png}

Further trials of rising amplitude give quicker onset of calcium dynamics, but overall current (the area of the colored regions) remains mostly constant.

Varying only the calcium-dependent potassium channel without activating the transient calcium channel has no effect, as no calcium is flowing in to activate the potassium channel.  Thus, it was added directly to the previous model with only a transient calcium current to see what the interactions were.  

The calcium current in some ways resembles the sodium current of the Hodgkin-Huxley model.  It is a positive ion current with a large concentration outside of the cell, which opening of the corresponding channels produces a depolarization.  The transient calcium current also has an inactivation state, much like the sodium channel, which occurs when the cell becomes adequately depolarized.  The main difference between them is their timescale, with the calcium channel acting over a much broader time period than the sodium current.  

In this way we would expect that the calcium-dependent potassium channel would act towards the calcium channel in the same manner as the voltage-gated potassium channel acts towards the sodium channel, balancing the depolarization of the influx of calcium by repolarizing the cell.  Since this potassium current works only in the presence of calcium, each set of dynamics can vary independently, even though potassium is involved in both.  

Here is the result of the combination of these two currents:

\includegraphics[scale=0.8]{kca.png}

In this series, potassium conductance is increased from 0.00345 to 0.0345 (one order of magnitude) in 20 trials over a period of 500 ms.  Membrane potential is relative to -75 mV, so really this starts at -70 mV and depolarizes down as far as -74 mV and as high as -67 mV.  At its base level (in blue), the calcium current and the potassium current balance in a long oscillation (period of almost 250 ms).  As potassium current increases, the period shortens until the oscillation is damped entirely.  

Ultimately, in conjunction with the original Hodgkin-Huxley model, these currents provoke a bursting of action potentials, followed by a period of silence.  This is due to the subtle depolarization provided by the transient calcium current, followed by the hyperpolarization of the calcium-dependent potassium current.  The neuron bursts while the calcium current holds sway, and then falls silent once the potassium current kicks in.  

This can be seen below:

\includegraphics[scale=0.6]{burst.png}

This is a single trial over a period of 50 ms.  The initial burst lasts only for the first 15 ms, after which it cannot overcome the hyperpolarizing influence of the calcium-dependent potassium current.

\end{document}
