\title{Systems Approach - Diagram Description}
\author{Ryan Spangler}
\date{\today}

\documentclass[11pt]{article}

\usepackage{commath}
\usepackage{graphicx}
\usepackage{listings}
\usepackage{amsfonts}

% python highlighting ----------
\usepackage{color}
\usepackage{listings}
\usepackage{textcomp}
\usepackage{setspace}
%\usepackage{palatino}

%\doublespacing

\setcounter{secnumdepth}{0}

\begin{document}
\maketitle

I approached this project as a guide for people who may have never even heard of systems before.  I tried to balance between clarity and density of information, leaning towards total simplicity as the ultimate goal.  In the process I labored to pare away all unnecessary detail and present each subject in the most straightforward way possible.  I assume nothing from the reader.  I am hoping that adults and kids alike can get something useful from it, as I worked hard to extract the essential systems ideas and present them in the clearest manner I was capable of.

The diagram is really to be used in conjunction with the document.  Once the document is read, the diagram can stand on its own, but I was not able to add descriptions of each phase directly into the diagram without unnecessarily complicating the presentation.  

I used the heuristic of ``something you would want to post on the wall'' as a guide to crafting the diagram.  I attempted to apply the systems approach I detail in the accompanying document and that is presented in the diagram in the construction of the diagram, as a proof of concept of sorts.  Enjoy!

\end{document}  

