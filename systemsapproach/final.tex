\title{Systems Approach - Final}
\author{Ryan Spangler}
\date{\today}

\documentclass[10pt]{article}

\usepackage{commath}
\usepackage{graphicx}
\usepackage{listings}
\usepackage{amsfonts}

% python highlighting ----------
\usepackage{color}
\usepackage{listings}
\usepackage{textcomp}
\usepackage{setspace}
%\usepackage{palatino}

%\doublespacing

\setcounter{secnumdepth}{0}

\begin{document}
\maketitle

Each individual perspective is crucial, but it is no one perspective that is the truth.  Somewhere, after all perspectives have been heard and gathered, they are examined as a whole, and a broader perspective that takes each into account is forged from the synthesis of these individual ones. 

There is a dynamic in a collective that is not present in the individuals.  When working as a team, if the individuals reach a point where they are predicting and supporting what one another will do, a tensegrity is formed that behaves as if there is a single collective action at work.

shared vision
team learning 
mental models
personal mastery


\end{document}  

