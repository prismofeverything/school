I remember when I first wondered what everything is. I was very young, and suddenly looked around myself and was overwhelmed by ``What is this? What is all of this? How is this happening?'' I was struck, and in a way, have never recovered. It was possibly my first real question, and I still have not answered it, even though seeking some kind of understanding of how existence can happen is what I spend all of my time doing. It is amazing to witness this one question decompose into and endless variety of smaller questions. It seems every possible question is implicated in this one deceptively simple question. Deceptively simple, because as I gawked at it that first time, it seemed the most basic, the *simplest* question I could ask.  Indeed, the only question. 

This question has led me through a large cross-section of subjects and activities, and as life goes on I come to see how much learning and understanding any particular thing contributes to understanding everything else. I am currently a programmer by trade, and find it a great medium for building and modeling any given process. Its pliability is rooted in its generality and composability, to the degree that I am truly limited only by imagination. Programming also has taught how to look at problems to extract the actual essence of what is happening. This applies to situations every day, and I find the practice of seeing things clearly is never complete, and requires constant attention. Though programming is the medium, I draw inspiration from every field I come across, focusing on cell biology, ecology, mathematics, music, neuroscience and physics. I have always deeply suspected that there is a relation between all of these things.  Every day new pieces are revealed to me, new connections are made, and I am constantly surprising myself. Cell biology I take the greatest attention to, as it studies the most basic system that we consider alive. It is the essence, nothing is extraneous. It has lessons in self-regulation, robustness and flexibility that could be applied to many currently baffling problems we currently face. There are a number of ways software could benefit from biologically inspired algorithms and processes, and the ways in which the systems that maintain the health of a cell apply to present day society could be considered pressing. 

I have always been driven and self-motivated, and since I graduated in 2005 I have been studying constantly. I didn't really have a name for what my interests were, as they included many fields which had names but seemed also to not be any one of them in particular. It cut across disciplines, while taking lessons from all of them. When I discovered there was this field called Systems Science I was overjoyed. There are others interested in these things as well, it turns out. Since then I have been reading everything that pertains to systems I could get my hands on, and have been looking at this Master's program for a number of years now. I would be thrilled at the opportunity to study the field I have realized is my passion. 

I may not answer that first question asked all those years ago (though I still hold out hope), but there are a number of questions that original question has spawned which I feel are more plausibly within reach. For example, how much of living systems can be modeled with programs? What else is necessary to realize truly living systems, if software is not adequate? Is there an essential process which defines life? How is this exhibited along all the various scales living things find themselves in, even down to the molecular and quantum level? Is our cognitive experience somehow distinct, or do all organisms experience experience to a greater or lesser degree? How do the individual actions of all of the various components of a system contribute to the activity of the whole, and how is the activity of the whole necessarily rooted in the coherent activity of its components? These are the ones that rattle off the top of my head, but there are many more. An infinitude, really, laid out in bas relief, forming more or less a reflection of existence itself. These are the issues I wish to pursue, and I will continue to pursue for my entire life. Perhaps one day I will be able to honor that original question with a suitable answer, and find the cycle complete, after a journey I never could have conceived of when the question first struck. A quote from T. S. Eliot comes immediately to mind, ``We shall not cease from exploration, and the end of all our exploring will be to arrive where we started and know the place for the first time.''
